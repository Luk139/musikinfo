\chapter{Fazit}
%

Um eine Trennung verschiedener Musikinstrumentgruppen durchzuführen, wird zuvor die Theorie bearbeitet.
Hierfür wird der Prozess von der Entstehung eines Tons bis zur Trennung einer Wav-Datei herausgearbeitet.
Dabei werden Alternativen zu dem vorgegebenen Vorgehen verglichen und evaluiert.
Es gibt einige alternative Speicherformen von Audio-Dateien (siehe: ...) und alternative Methoden zur Transformation einer Audio-Datei (Synonym?) (siehe: ...).

\par

Anschließend wird die behandelte Theorie anhand einer praktischen Durchführung wiedergegeben.
Die Ausarbeitung der Theorie und das erworbene Verständnis für das Themengebiet helfen bei der praktischen Umsetzung.
Damit wird es Leser:innen ermöglicht einen Basiscode zu implementieren, den sie nachvollziehen und darauf aufbauen können.
Außerdem können durch die Umsetzung des Codes die eingangs gestellten Forschungsfragen und Hypothesen (Kapitel fehlt noch?) beantwortet werden.

%
\section{Beantwortung der Forschungsfrage}
%

Durch die Erarbeitung von verbundenem Hintergrundwissen wird ein Algorithmus zur Beantwortung der Forschungsfrage implementiert.
Dieser bestätigt, dass eine Wav-Datei mit perkussiven und melodischen (richtig?) Instrumenten in die jeweiligen Gruppen trennbar ist.
Der Programmcode und das Ergebnis einer erfolgreichen Durchführung auf eine Wav-Datei werden an diese Arbeit angehangen (wie?).
Zudem werden eingangs gestellte Fragen beantwortet und die Hypothesen behandelt.

%
\textbf{Beantwortung von Unterfragen}
%

In \cref{sounds} wird erläutert wie ein Ton entsteht und dessen Darstellung in Wellenform aussieht (siehe: \cref{wav_sound}).
Anhand der Anzahl der Anzahl an Wellen pro Sekunde wird die Frequenz in Hertz berechnet.
Bei der Analyse mehrerer Töne wird es schwieriger einzelne Signale aus einer Wellenform abzulesen (siehe: \cref{wav_multiple_sounds}).

\par

In diesem Projekt werden ausschließlich akkustische Instrumente verwendet, da diese aufgrund der natürlichen Entstehung effektiver von der Fourier Transform zugeordnet werden können.
Dabei werden lediglich perkussive von harmonische Instrumente (jeweils Teil der akkustischen Instrumente) getrennt, da deren unterschiedliche Ausschläge differenzierbar sind.
Allerdings konnte im Rahmen dieses Projekts die Wahl der Musikinstrumentgruppen nicht bewiesen, sondern lediglich recherchiert und referenziert werden.

\par

Es wurden WAV, MP3, WMA, AAC, OGG, FLAC und RM als verbreitete Audio-Speicherformen herausgearbeitet.
Dabei wurden Wav-Dateien als geeignetste Speicherform für dieses Projekt ermittelt, aufgrund des geringen Speicherverlusts.
Ebenso hat sich die Fourier Transform als präzisiste Transformation im Gegensatz zur Wavelet Transform herausgestellt.
Während des Projekts wurde die Short-Time Fourier Transform (STFT) verwendet, die auf der Fourier Transform aufbaut.

%
\section{Kritische Reflexion der eigenen Arbeit}
%

Bei der Umsetzung des Codes werden lediglich Zusammenhänge und nicht einzelne Befehle erläutert.
Um den Algorithmus nachzuvollziehen, werden Vorkenntnisse im Programmieren benötigt.
Für das weiterarbeiten am Programmcode werden Python-Kenntnisse benötigt (oder?).
Allerdings wird für ein Verständnisder behandelten Theorie kein Vorwissen benötigt.

\par

Es werden lediglich vorgegebene Gruppen von Musikinstrumenten getrennt.
Die Trennung einzelner Musikinstrumente oder anderer Instrumentgruppen kann der Algorithmus nicht durchführen (oder?).
Damit erfüllt der Algorithmus nur einen speziellen Anwendungsfall von den möglichen Instrumenten die getrennt werden.

\par

Das Ergebnis des Projekts ermöglicht ausschließlich dei Verwendung von Wav-Dateien.
Diese werden aufgrund des hohen Speicherbedarfs selten verwendet und schwierig zu erhalten.
Bei der Anwendung des Algorithmus auf ein gewähltes Lied, ist es schwierig dieses als Wav-Datei zu finden.

\par

Zudem basiert die Wahl der trennbaren Musikinstrumentgruppen und verwendeten Speicherform lediglich auf einer eingangs gestellten Hypothese.
Diese Hypothese kann mittels Quellen und erläuterter Zusammenhänge bekräftigt werden.
Allerdings werden diese Thesen nicht bewiesen durch eine Umsetzung möglicher Alternativen.

%
\section{Ausblick für die Zukunft}
%

In der Zukunft bieten sich vielfältige Möglichkeiten zur Weiterentwicklung dieses oder ähnlicher Algorithmen, insbesondere im Hinblick auf komplexere Anwendungsfälle. Ein bedeutender Fortschritt wäre die Trennung einzelner Musikinstrumente, anstatt lediglich harmonische und perkussive Frequenzen zu unterscheiden. So könnten beispielsweise Instrumente wie Bass, Gitarre oder Klavier sowie Schlagzeug und Cajón isoliert werden. Auch komplexere Klangstrukturen, wie gelayerte Synthesizer, könnten mit optimierten Algorithmen analysiert und aufgelöst werden.

Ein solches System würde Musiker:innen enorme Vorteile bieten. Sie könnten die einzelnen Noten ihres Instruments gezielt herausfiltern und zum Üben oder Lernen verwenden, ohne zusätzliche Kosten für Notenmaterial oder separate Audiotracks. Zudem könnte es möglich werden, nicht transkribierte Musikstücke automatisch zu analysieren und in Notenform zu übertragen. Ein weiteres Anwendungsbeispiel wäre die Erstellung von Playbacks: Ein Schlagzeugtrack könnte beispielsweise entfernt werden, um darauf basierend eine eigene Aufnahme zu erstellen.

Der bestehende Algorithmus könnte zudem durch die wiederholte Anwendung mit angepassten Filtern auf die bereits getrennten WAV-Dateien verfeinert werden, um noch spezifischere Ergebnisse zu erzielen. Darüber hinaus bieten moderne Ansätze mit künstlicher Intelligenz eine vielversprechende Möglichkeit, die Erfolgsquote bei der Trennung und Analyse weiter zu erhöhen.

Eine Erweiterung des Algorithmus auf andere Audioformate wie MP3 wäre ebenfalls sinnvoll, um die Ergebnisse auch auf verbreitetere und leichter zugängliche Dateien anzuwenden. Trotz des potenziellen Qualitätsverlusts könnte dies die Nutzbarkeit des Systems für eine breitere Zielgruppe steigern. Ein abschließender Vergleich der Ergebnisse zwischen WAV- und MP3-Dateien würde zusätzlich wertvolle Erkenntnisse über die Leistungsfähigkeit des Algorithmus liefern.

Langfristig eröffnet diese Forschung Perspektiven, um neue Technologien für Musiker:innen, Produzent:innen und Musikliebhaber:innen zugänglicher zu machen, während gleichzeitig der Grundstein für weiterführende Innovationen gelegt wird.
