\chapter{Stand der Wissenschaft}
\label{stand_der_wissenschaft}

Die Trennung von Musikinstrumenten ist ein praktischer Anwendungsfall zur Separation von Signalen. Es entwickelte sich aus dem Gebiet der Sprachverarbeitung und übernimmt einige Techniken \parencite{Mueller_2011}. Dabei liegen in einer Aufnahme mehrere Signale in Form von einer Funktion (hier: Wellenform) vor. In der Wellenform sind die einzelnen Signale schwierig zu identifizieren.

\par

Bei dem Vorgehen zur Trennung von Signalen wird eine Funktion von der Abhängigkeit zur Zeit in die Abhängigkeit zur Frequenz umgeformt. Anhand der Frequenz können unterschiedliche Signale identifiziert werden. Unter anderem werden die Frequenzen unterschiedlichen Instrumenten oder Musiknoten zugeordnet.

\par

Inzwischen wurden mehrere Methoden zur Transformation einer Funktion in Abhängigkeit zur Zeit in die Abhängigkeit zur Frequenz entwickelt. Diese unterscheiden sich in der Darstellung der Funktion und der Genauigkeit des Ergebnisses.

\par

In diesem Projekt wird die Fourier Transform angewendet, die als verbreitet und effektiv gilt \parencite{fourier_transform_importance}. Es wird überprüft, ob die Implementierung und Anwendung ohne Vorkenntnisse und anspruchsvolle Hardware umsetzbar ist. Eine weitere verbreitete Methode ist die Wavelet Transform \parencite{Guo_2022}. Diese wird in \cref{wavelet-transformation} erläutert, um die Entscheidung für die Fourier Transform zu begründen.

% Zusammenfassung der wichtigsten Forschungsergebnisse zu einem bestimmten Thema

% aktueller Stand deines Forschungsfeldes. Du zeigst mit ihm auch auf, an welchem Punkt deine Fragestellung ansetzt.

% https://www.scribbr.de/aufbau-und-gliederung/forschungsstand/