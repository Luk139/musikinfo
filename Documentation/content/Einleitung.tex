\chapter{Einleitung}
%

Musik ist zu einem Teil des täglichen Labens vieler Menschen geworden. Durch den Konsum und den wirtschaftlichen Ertrag wird an der Produktion und Analyse von Musik geforscht (Quelle?).

\par

Ein Teil der Forschung bezieht sich auf die Trennung einer Wellenform in die unterschiedlichen Funktionen der Frequenzen. Zuvor beinhaltet die Wellenform eine oder mehrere Sinus- und Kosinusfunktionen, die eine neue Funktion bilden. Dies wird sowohl auf audiovisuelle als auch auf visuelle Funktionen angewendet. Beispielsweise kann die Funktion mehrere Instrumente beinhalten, die anhand der Funktion nicht identifizierbar sind. Einer der Algorithmen zur Trennung von Signalen heißt Fourier Transform und wird in diesem Projekt behandelt.

\par

Unter anderem werden Musikinstrumente in Liedern getrennt und einzeln angehört oder wiederverwendet. In diesem Projekt werden zum Einstieg lediglich perkussive aus akkustischen Instrumenten getrennt. Zwischen den jeweiligen Instrumentengruppen wird in \cref{sounds} unterschieden.

%
\section{Relevanz}
%

Es gibt unterschiedliche Kontexte, in denen die Trennung von Audiosignalen zum Einsatz kommt. Häufig ist eine Eingabefunktion als Summe aller Signale schwierig zu analysieren oder weiterzuverarbeiten. Beispielsweise, wenn einzelne Instrumente im Nachhinein bearbeitet werden sollen.

\par

In Zeiten von zunehmend digital produzierter Musik und Verbreitung von Informatik stellt sich die Frage, wie Musiksignale digital aufgebaut sind und wie man mit ihnen arbeiten kann. Im Kontext dieser Arbeit wird die Fourier Transform verwendet, um eine Tondatei in Gruppen von harmonischen und perkussiven Musikinstrumenten zu zerlegen. Dies kann mit weiterer Modifikation verwendet werden, um bestimmte Instrumente zum üben zu trennen oder Hintergrundgeräusche auszublenden.

\par

% Außerdem werden in dem Projekt die Überprüfung der Funktionalität einer Ampel oder das stimmen von Musikinstrumenten in Echtzeit behandelt. Dies sind nur Beispiele dafür, welche Anwendungsfälle es für die Fourier Transform gibt.

(Etwas zu kurz?)

%
\section{Vorgehen}
%

Ohne jegliches Vorwissen, wie die Signale einer Musikdatei gespeichert und von einem Computer interpretiert oder getrennt werden, wird das Projekt durchgeführt. Die Erkenntnisse und notwendiges Hintergrundwissen werden dokumentiert und erklären die Logik und Zusammenhänge der Fourier Transform. Durch die Dokumentation wird anderen Leser:innen ein erstes Verständnis von WAV-Dateien vermittelt auf dem im Anschluss aufgebaut werden kann.

\par

(WAV-Dateien sehr spezifisch: Den Leser:innen wird auch die Funktionalität der FFT und Code etc. vermittelt?)

\par

Das Projekt dokumentiert anschließend die Implementierung eines Algorithmus zur Trennung der harmonischen und perkussiven Tongruppen einer WAV-Datei mittels Fourier Transform, sowie das schreiben auf zwei getrennte WAV-Dateien mittels Inverse Fourier Transform (siehe: \cref{Code}). Dabei wird zuvor behandelte Logik und Hintergrundwissen referenziert und in der Praxis erklärt.