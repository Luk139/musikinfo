\chapter{Einleitung}
%

Musik ist zu einem Teil des täglichen Labens vieler Menschen geworden. Durch den Konsum und den wirtschaftlichen Ertrag wird an der Produktion und Analyse von Musik geforscht (Quelle?). Ein Teil der Forschung bezieht sich auf die Trennung einer Wellenfunktion in die unterschiedlichen Funktionen der Frequenzen. Dies wird sowohl auf audiovisuelle als auch auf visuelle Funktionen angewendet. Eine der Algorithmen zur Trennung von Signalen heißt Fourier-Transformation und wird in diesem Projekt behandelt.

\par

Unter anderem werden Musikinstrumente in Liedern getrennt und einzeln angehört oder wiederverwendet.

%
(Frequenzen und Wellenfunktion erklären?)
%

%
\section{Relevanz}
%

Es gibt unterschiedliche Kontexte in der die Trennung von Signalen durchgeführt wird. Häufig ist eine Eingabefunktion als Summe aller Signale schwierig zu analysieren oder weiterzuverarbeiten.

\par

In Zeiten von zunehmend digital produzierter Musik und Bedeutung von Informatik stellt sich die Frage, wie Musiksignale aufgebaut sind und wie man mit ihnen arbeiten kann. Im Kontext dieser Arbeit wird die Fourier-Transformation verwendet, um eine Tondatei in verschiedene (Gruppen von?) -> (akkustik und percussive (deutsch)) Musikinstrumente zu zerlegen. Dies kann verwendet werden, um bestimmte Instrumente zum üben zu trennen oder Hintergrundgeräusche auszublenden.

\par

Außerdem werden in dem Projekt die Überprüfung der Funktionalität einer Ampel oder das stimmen von Musikinstrumenten in Echtzeit behandelt. Dies sind nur Beispiele dafür, welche Anwendungsfälle es für die Fourier Transformation gibt.

%
\section{Vorgehen}
%

Ohne jegliches Vorwissen, wie die Signale einer Musikdatei gespeichert und von einem Computer interpretiert oder getrennt werden, wird das Projekt durchgeführt. Die Erkenntnisse und notwendiges Hintergrundwissen werden dokumentiert und erklären die Logik und Zusammenhänge der Fourier-Transformation. Durch die Dokumentation wird anderen Lesern:innen ein erstes Verständnis von WAV-Dateien vermittelt auf dem im Anschluss aufgebaut werden kann.

\par

Das Projekt dokumentiert anschließend die Implementierung eines Algorithmus zur Trennung einer WAV-Datei mittels Fourier Transformation von Musikinstrumenten, sowie das schreiben auf zwei getrennte WAV-Dateien mittels Inverse Fourier Transform (siehe: ...?). Dabei wird zuvor behandelte Logik und Hintergrundwissen referenziert und in der Praxis erklärt.