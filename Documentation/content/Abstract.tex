\chapter*{Kurzfassung/\emph{Abstract}}
\label{chap:abstract}
%

Diese Arbeit untersucht die Machbarkeit der Trennung perkussiver und melodischer Frequenzen in einer Wav-Datei.
Im Rahmen des Projekts wird die Fourier-Transformation als Kernmethode zur Signalverarbeitung eingesetzt, um harmonische und perkussive Bestandteile von Audiosignalen zu analysieren und zu extrahieren.
Ergänzend wird auf alternative Methoden wie die Wavelet-Transformation eingegangen, um deren Vor- und Nachteile gegenüber der Fourier-Transformation zu bewerten.

\par

Die praktische Umsetzung erfolgt mit Python und der Bibliothek Librosa.
Die entwickelte Methode ermöglicht es, Audiosignale in getrennte Komponenten zu zerlegen, was vielfältige Anwendungen in der Musikproduktion, Audioanalyse und -restauration eröffnet.
Es wird gezeigt, dass Wav-Dateien aufgrund ihres geringen Informationsverlusts besonders geeignet sind, während MP3-Dateien aufgrund ihrer Verfügbarkeit ein zukünftiges Ziel für Optimierungen darstellen könnten.
Abschließend reflektiert die Arbeit die Limitationen des Ansatzes und gibt einen Ausblick auf mögliche Erweiterungen, einschließlich der Nutzung maschinellen Lernens für komplexere Signaltrennungen.