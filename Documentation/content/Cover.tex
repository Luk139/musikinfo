\begin{titlepage}
%
\sffamily% Umschalten auf serifenlose Schrift
%
\begin{center}
\begin{tikzpicture}
 \fill[THRed] (0, 0) rectangle (\textwidth/3, 3pt);
 \fill[THOrange] (\textwidth/3, 0) rectangle (2*\textwidth/3, 3pt);
 \fill[THPurple] (2*\textwidth/3, 0) rectangle (\textwidth, 3pt);
\end{tikzpicture}
\end{center}
%
\vfill
%
\begin{huge}
Umsetzbarkeit der Trennung perkussiver und harmonischer Frequenzen in einer Wav-Datei\\[10mm]
\end{huge}
%
Projektteil der Belegung einer Wahlspezialisierung\newline
%\emph{Master/Bachelor of Arts/Engineering/Laws/Science}\newline
im Studiengang Informatik\newline
an der Fakultät für Informatik und Ingenieurwissenschaften\newline
der Technischen Hochschule Köln
%
\vfill
%
\begin{tabular}{@{}ll}
vorgelegt von: & Lukas Fey, Nicolas Friedmann\\
Matrikel-Nr.:  & 123 456 789, 111 55 463\\
Adresse:       & Neusser Str 31, Auf der Platte.~1\\
               & 40667 Meerbusch, 51643 Gummersbach\\
               & lukas.fey@smail.th-koeln.de, nicolas-friedmann@gmx.de\\[5mm]
eingereicht bei:   & Prof. Dr. Lutz Köhler\\
\end{tabular}	
%
\\[10mm]
%
Gummersbach, 12.01.2025%
%
\rmfamily% Umschalten auf Standard-Schrift mit Serifen
%
\end{titlepage}